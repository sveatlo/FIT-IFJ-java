\documentclass[12pt]{article}

\usepackage[utf8]{inputenc}
\usepackage[slovak]{babel}
\usepackage[letterpaper, margin=2cm]{geometry}




\title{Dokumentácia projektu IFJ}
\author{Svätopluk Hanzel}
\date{09.12.2016}

\begin{document}
	\begin{titlepage}
		\begin{center}
		{\scshape\LARGE Vysoké učení technické v Brně \par}
		{\Large Fakulta informačních technologií\par}


		\vspace{3cm}

		{\scshape\LARGE Dokumentácia projektu IFJ\par}
		{\Large Interpret jazyka IFJ16 \par}


		\vfill

		Svätopluk Hanzel (xhanze10) - vedúci\\
		Matúš Juhász (xjuhas02)\\
		Dominik Križka (xkrizk02)\\
		Tomáš Haas (xhaast00)\\
		Samuel Hulla (xhulla00)\\
		\vspace{1cm}

		{\hfill Brno, 22.12.2016}
	\end{center}
	\end{titlepage}
	\tableofcontents{}
	\newpage

	\section{Úvod}
		Táto dokumentácia slúži ako textová časť riešenia projektu z predmetu IFJ pre rok 2016/2017. Celý dokument sa delí do niekoľkých kapitol a príloh. Postupne sú rozobrané všetky dôležié prvky implementácie jednotlivých častí interpretu.
		\newpage
	\section{Práca v tíme}
        \subsection{Komunikácia v tíme}
	        Zo začiatku projektu bola v našom tíme snaha o pravidelne stretnutia 1-2-krát týždenne. To sa však postupne ukázalo ako neefektívne vzhľadom na malé množstvo užitočných informácii, ktoré sa prebrali počas týchto stretnutí a tak sa celá komunikácia presunula na Slack a skôr sa rozdelila medzi ľudí spolupracujúcich na jednotlivých častiach projektu.\\
	        Rozhodnutia o smerovaní projektu boli síce centralizované, ale zároveň otvorené k diskusii.
        \subsection{Verzovací systém}
			Ako nástroj na správu verzií a zdielanie kódu sme zvolili GIT, pričom vzdialený ako vzdialený repozitár bol zvolený privátny server GitLab, ktorý zároveň slúžil na zadávanie tzv. , teda problémov, ktoré treba vyriešiť.
		\subsection{Rozdelenie práce}
        \subsection{Automatizované testy}
	\section{Riešenie interpretu}
		\subsection{Lexikálna analýza}
		\subsection{Syntaktická a sémantická analýza}
		\subsection{Interpretácia}
		\subsection{Algoritmy}
			\subsubsection{Knuth-Morris-Prattův algoritmus}
				Jedná se o algoritmus pro vyhledání podřetězce v řetězci. Klasické (triviální) vyhledávání podřetězce – tedy když postupně procházíme řetězec a u každého znaku zjišťujeme, zda se shoduje s podřetězcem, a v případě neshody se posuneme na další znak a musíme porovnávat s podřetězcem zase od začátku – vede v nejhorším případě až na časovou složitost \textbf{O(m·n)}, kde \textbf{m} je délka řetězce a \textbf{n} délka hledaného podřetězce. Tento algoritmus je tedy velmi neefektivní. O(m+n), tedy je mnohem rychlejší. KMP algoritmus dokáže sledovat, zda se v hledaném podřetězci opakují určité skupiny znaků (podřetězce). Při procházení textu pak při neshodě zjistí, zda se v části podřetězce, která už byla zkontrolována (dokud nedošlo k neshodě), nachází sufix, který je shodný s prefixem.\\
				\textbf{Př.:}\\
				Hledaný podřetězec je \textbf{abcxyabcijk}. Uvažujme, že při procházení textu nastane neshoda až na znaku \textbf{c}. Část podřetězce, která podmínkou prošla, je tedy \textbf{abcxyabc}. V této části je skupina znaků \textbf{abc}, která se opakuje, a to tak, že je zároveň jejím prefixem i sufixem.
				Pokud se v podřetězci nachází taková skupina znaků, při neshodě se pak nemusí začínat znovu od začátku podřetězce a v řetězci se nemusíme vracet zpět ke znaku, od kterého jsme začínali kontrolovat. Protože (opět uvažujme uvedený příklad) pokud jsme prošli podřetězcem bez problému až po \textbf{i}, jsme si jistí, že v procházeném textu jsou předchozí tři znaky \textbf{abc}, což se shoduje s prvními třemi znaky, které bychom hledali, pokud bychom procházeli podřetězec znovu od začátku. Můžeme tedy tyto tři znaky přeskočit a pokračovat dále.
				Nejdříve je nutné projít podřetězec a vytvořit pole (v naší funkci pojmenované fail), do kterého se zapíšou hodnoty podle toho, kolika-znakové skupiny se v podřetězci opakují. Tyto hodnoty pak určují, na kterou pozici v podřetězci se musíme vrátit v případě neshody (tedy i kolik znaků v řetězci již můžeme přeskočit).\\
				\textbf{Př.:}\\\\
				\begin{center}
					\begin{tabular}{|c|c|c|c|c|c|c|c|}
						\hline 
						a & b & c & d & a & b & c & a \\ 
						\hline 
						0 & 0 & 0 & 0 & 1 & 2 & 3 & 1 \\ 
						\hline 
					\end{tabular}\\
				\end{center}
				Algoritmus KMP pak prochází postupně řetězec a porovnává jednotlivé znaky se znaky podřetězce. V případě neshody se pak podle hodnoty v tabulce fail posune na určitou pozici v podřetězci a znaky před touto pozicí již nemusí znovu kontrolovat.
				Výsledkem je algoritmus pracující se složitostí \textbf{O(m+n)}, jelikož při porovnání textu se algoritmus nikdy nevrací, celý ho tedy projde se složitostí O(m), ale předtím je nutné tzv. Předzpracování – vytvoření tabulky fail, což znamená projít podřetězec se složitostí O(n).
				
			\subsubsection{Heapsort algoritmus}
				Řazení haldou/hromadou je velmi efektivní řadící algoritmus, který pracuje s časovou složitostí \textbf{O(n·logn)}, která je však zaručená. Jeho použití je tedy někdy vhodnější než použití quicksortu, který může být v některých případech rychlejší, ale v nejhorších případech dosahuje až složitosti \(O(n^2)\).
				Tento algoritmus řadí pole, které má strukturu binární hromady. Hromada je struktura stromového typu, pro kterou platí, že mezi otcovským uzlem a všemi jeho synovskými uzly je vždy stejná relace (např. otec vždy větší než jeho synové). Binární hromada je založena na binárním stromu.
				Pro hromadu implementovanou jako pole (tedy v našem případě řetězec) platí, že otcovský uzel na indexu i má vždy levého syna na indexu \textbf{2i} a pravého syna na indexu \textbf{2i+1}.
				V každém kroku řazení dojde k porušení struktury a je potřeba se stromem tzv. „zatřást“ (shiftdown), čímž se struktura hromady obnoví.
				Tím se dostane na vrchol hromady prvek podle určitého pravidla (např. největší nebo nejmenší). Pokračujeme tak, že vrchol haldy „utrhneme“ a vložíme ho na další pozici již seřazeného řetězce. Poté pracujeme s haldou o jeden prvek menší, musíme ji znovu opravit (pomocí siftdown) a tento postup opakujeme, dokud má halda nějaké prvky.
				Výsledkem je algoritmus se složitostí \textbf{O(n·logn)}. Operace shiftdown, neboli rekonstrukce hromady, totiž pracuje se složitostí O(logn), je tedy schopná rychle najít extrém v daném poli. Heapsort je algoritmus nestabilní, což znamená, že může dojít k prohození prvků se stejnou hodnotou a že dochází k přesouvání prvků velkými skoky, a je nepřirozený, což znamená, že nehraje žádnou roli fakt, zda bylo pole před začátkem řazení již částečně seřazené.
				
			\subsection{Tabulka symbolů}	
				Je tvorená pomocou binárného vyhladávacieho stromu. Jeho výhodou je, že sú v ňom prvky zoradené tak, že klúče všetkých uzlov lavého podstromu 
				sú menšie ako klúč uzlu a klúče pravého podstromu sú väčšie ako klúč uzlu. To ulahčuje vyhladávanie a nájdeniu prvku je rýchlejšie. 
				Nevýhodou je, že musíme prvky udržovat usporiadané, to znamená, že ak chceme vložit nový prvok, musí být vložený presne tam, kam podla 
				svojej velkosti patrí.
				
				V binárnom strome máme uložené názvy symbolou ich data, ich typ a konkrétne id. Naša implementácia je napísaná rekurzívne. Pre vkladanie nových symbolov sa využíva funkcia \textit{table insert symbol}. Ktorá využíva funkciu \textit{tree insert} pre prácu so binárnym stromom sa využíva aj funkcia \textit{tree search} ktorý vyhladáva na základe zadaného kluča.Po skončení práce sa pomocou funkcie \textit{tree dispose} . Pre naše potreby sme potrebovali hlbkovú kopiu nášho binárneho stromu a z toho dôvodu vznikla funkcia \textit{tree copy} .
	\section{Prílohy}
		\subsection{Štruktúra lexikálneho analyzátora}
		\subsection{LL gramatika}
		\subsection{Algoritmy}

\end{document}
